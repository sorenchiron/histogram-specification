% XeLaTeX can use any Mac OS X font. See the setromanfont command below.
% Input to XeLaTeX is full Unicode, so Unicode characters can be typed directly into the source.

% The next lines tell TeXShop to typeset with xelatex, and to open and save the source with Unicode encoding.

%!TEX TS-program = xelatex
%!TEX encoding = UTF-8 Unicode

\documentclass[12pt]{article}
\usepackage{geometry}                % See geometry.pdf to learn the layout options. There are lots.
\geometry{letterpaper}                   % ... or a4paper or a5paper or ... 
%\geometry{landscape}                % Activate for for rotated page geometry
%\usepackage[parfill]{parskip}    % Activate to begin paragraphs with an empty line rather than an indent
\usepackage{graphicx}
\usepackage{amsmath,amsthm,amssymb}
\usepackage{subfigure}
\usepackage{amsfonts}
\usepackage{xcolor}
\def\d{\mathrm{d}}
\def\E{\mathrm{E}}
\def\L{\mathcal{L}}
\newcommand*\diff{\mathop{}\!\mathrm{d}}
% Will Robertson's fontspec.sty can be used to simplify font choices.
% To experiment, open /Applications/Font Book to examine the fonts provided on Mac OS X,
% and change "Hoefler Text" to any of these choices.

\usepackage{fontspec,xltxtra,xunicode}
\defaultfontfeatures{Mapping=tex-text}
\setromanfont[Mapping=tex-text]{Hoefler Text}
\setsansfont[Scale=MatchLowercase,Mapping=tex-text]{Gill Sans}
\setmonofont[Scale=MatchLowercase]{Andale Mono}

\title{Histogram Specification}
\author{CHEN Siyu}
%\date{}                                           % Activate to display a given date or no date

\begin{document}
\maketitle

% For many users, the previous commands will be enough.
% If you want to directly input Unicode, add an Input Menu or Keyboard to the menu bar 
% using the International Panel in System Preferences.
% Unicode must be typeset using a font containing the appropriate characters.
% Remove the comment signs below for examples.

% \newfontfamily{\A}{Geeza Pro}
% \newfontfamily{\H}[Scale=0.9]{Lucida Grande}
% \newfontfamily{\J}[Scale=0.85]{Osaka}

% Here are some multilingual Unicode fonts: this is Arabic text: {\A السلام عليكم}, this is Hebrew: {\H שלום}, 
% and here's some Japanese: {\J 今日は}.

\begin{abstract}
This article describes Histogram Specification, whose purpose is to obtain a function $f(p)$ to map the given data into a specified histogram.

This is useful when raw images directly read from sensors are not ideal in brightness or contrast.
\end{abstract}


\section{Problem definition}

Given a series of data $P=\{p_1,p_2,...,p_N\}, p \in \mathbb{N}$, its histogram can be expressed as 
\begin{equation}
h_P(p)=n_p,
\end{equation}
in which, $n_p$ denotes the number of element that values $p$ in sequence $P$.

Given a desired target histogram $h_Q(q) = n_q$, the objective is to find a monotonic mapping function $f(p) = q$ such that:

\begin{align}
P'&= \{f(q_1),f(q_2\},...,f(q_N) \} = \{p'_1,p'_2,...,p'_N\}, \\
h_{P'}(p) &=h_{Q}(p). 
\end{align}

\section{Derivations}
The former problem definition is in discrete form. In some extreme but not rare cases, optimal solution may not exist for discrete histogram specification. For example:
\begin{align}
P=\{1,1,1,1,1\}, \\ 
Q=\{1,2,3,4,5\}.
\end{align}
In the above scenario, a desired $f(p)$ does not exist.

To simplify the problem, we assume continuous distribution on set $P$ and $Q$ such that $P,Q\in [0,1]^N$. In other word, $h_P(x)$ and $h_Q(x)$ are both continuous. 

According to the definition of the problem, several constrains hold as listed below:

\begin{align}
h_{P'}(x) &=h_{Q}(x),\label{eq:eqden} \\
\frac{\partial f(x)}{\partial x} & \ge 0,
\end{align}

in which $x \in [0,1] $.

According to Eq. \ref{eq:eqden}, if the distribution density is identical everywhere, then the cumulative distribution shall also be equal-valued.

\begin{align}
\int_0^x h_{P'}(z) \diff z &= \int_0^x h_{Q}(z) \diff z.
\end{align}

The above equation can be transformed into the following form:

\begin{align}
\int_0^{f(x)} h_{P}(z) \diff z &= \int_0^x h_{Q}(z) \diff z, \\
H_P(f(x)) &= H_Q(x),
\end{align}
where the original functions $H_P(x),H_Q(x)$ are the cumulative intensity functions of $h_P(x),h_Q(x)$. To obtain $f(x)$, the inverse function of $H$ can be applied to both side:

\begin{align}
H^{-1}_P \left( H_P(f(x)) \right)  &= H^{-1}_P \left( H_Q(x) \right).
\end{align}

For cumulative intensity function $H(x) \in [0,1], x\in [0,1] $, the inverse function is simply $H^{-1} = 1-H$. Hence the equation can be further substituted into:

\begin{align}
f(x) = 1 - H_P(H_Q(x)),
\end{align}

which is the desired optimal solution.

For discrete distributions $P,Q \in \mathbb{R}^N$ with $p_{max} = \max (P)$ and $q_{max}$ respectively, the corresponding mapping function is: 

\begin{align}
f(x) = q_{max} - \frac{q_{max}}{N}H_P(\frac{p_{max}}{N}H_Q(x \frac{ q_{max}}{p_{max}})),
\end{align} 


\section{Histogram Equalization}
A mapping function $f(x)$ that equalizes the histogram can be computed given that:
\begin{align}
h_Q(x) = C,
\end{align}
where $C$ is a constant value. 

Under continuous distribution assumption, we have:

\begin{align}
H_Q(x) &= x, \\
f(x) & = 1 - H_P(x).
\end{align}

Then in the discrete case, the mapping is :
\begin{align}
f(x) =  q_{max} - \frac{q_{max}}{N}H_P(x).
\end{align}





















\end{document}  